\documentclass[12pt]{article}
\usepackage{geometry}
\usepackage{titlesec}
\usepackage{fontsize}
\usepackage{setspace}
\usepackage{listings} % 用于代码展示

\renewcommand{\normalsize}{\fontsize{14}{17}\selectfont}
\setlength{\parskip}{0pt} % 段落间距设为 0
\setlength{\parindent}{1.27cm} % 强制首行缩进

% 页面边距设置
\geometry{a4paper, left=1in, right=1in, top=1in, bottom=1in}

% lstlisting 设置
\lstset{
  basicstyle=\ttfamily\fontsize{12}{14}\selectfont, % 设置字体大小为 12pt,行间距为 14pt
  breaklines=true,                                  % 自动换行
  breakatwhitespace=true,                           % 在空白处换行
  columns=flexible,                                 % 修复字符间距问题
  frame=single,                                     % 添加边框
  numbers=left,                                     % 显示行号
  numberstyle=\fontsize{10}{12}\selectfont,         % 设置行号字体大小为 10pt,行间距为 12pt
  showstringspaces=false                            % 不显示空格标记
}

% Section 样式
\titleformat{\section}[block]
  {\normalfont\fontsize{25}{22}\bfseries\centering}
  {}{0em}{}[\titlerule]

% Subsection 样式
\titleformat{\subsection}[block]
  {\normalfont\fontsize{18}{20}\bfseries\centering}
  {}{0em}{}[\titlerule]

% Title 设置
\title{Bomberman Game User Guide}
\author{}
\date{}

\begin{document}

\makeatletter
\renewcommand{\maketitle}{
  \begin{center}
    {\fontsize{36}{40}\bfseries \@title \par} % 主标题:字体大小 36pt,行间距 40pt,加粗
    \vskip 1em % 标题与正文的间距
    {\fontsize{14}{18} \@author \par} % 作者:字体大小 14pt,行间距 18pt
    \vskip 1em
    {\fontsize{12}{14} \@date \par} % 日期:字体大小 12pt,行间距 14pt
  \end{center}
}
\maketitle

\section*{How to start}
\hspace{1.27cm}Open bomberman/big-bang.rkt and click run.

\section*{Homepage}
\hspace{1.27cm}Press ``space'' key to enter the main page.

\section*{Movement}
\hspace{1.27cm}Player 1: The orange cat at the left-top corner. \\
\indent Use ``up'', ``down'', ``left'', ``right'' keys to move up, down, left, right. Use ``space'' key to put the bomb.

\indent Player 2: The brown dog at the right-down corner. \\
\indent Use ``w'', ``s'', ``a'', ``d'' keys to move up, down, left, right. Use ``g'' key to put the bomb.

\section*{Cell}
\hspace{1.27cm}\textbf{Christmas tree:} Can be destroyed by the bombs; players cannot pass; cannot put the bombs on it. After getting destroyed, becomes Snow Field.

\indent \textbf{Snowman stone:} Cannot be destroyed by the bombs; players cannot pass; cannot put the bombs on it.

\indent \textbf{Snow field:} Cannot be destroyed; walkable; can put the bomb on it. After the bombs explode, becomes Explosion Area.

\indent \textbf{Cell with bomb:} When players are currently on a cell with a bomb, they can still move to another walkable (Snow) field. However, players cannot move onto a cell that already has a bomb.

\section*{Bomb}
\hspace{1.27cm}\textbf{Time:} Put the bombs on the Snow Field, after 3s they will explode. The explosion lasts for 2s. \\
\indent \textbf{Range:} The normal bomb range is a symmetric cross field with 9 cells, which has a center and spreads 2 cells in every direction (up, down, left, right). If there is a Christmas Tree, the bomb destroys the Christmas Tree, and the explosion range stops in that direction. If there is a Snowman Stone, the explosion range stops in that direction. \\
\indent \textbf{Number of Bombs:} At the beginning, each player can put a maximum of 3 bombs, then every 30s, 1 bomb is added.

\section*{Round-Timer}
\hspace{1.27cm}\textbf{Total time:} 120s.

\section*{How to Play It}
\hspace{1.27cm} You move around on the layout, put the bombs on the Snow Field, destroy the obstacles (Christmas Tree), and try to kill the opponent. \\
\indent Avoid yourself to be in the Explosion Area. \\
\indent Let the opponent be inside Explosion Area to kill it. \\
\indent You can also be killed by the bombs that you put.

\section*{Game Ending Conditions}
\hspace{1.27cm}\textbf{Player2 win:} Player1 is in the Explosion Area. \\
\indent \textbf{Player1 win:} Player2 is in the Explosion Area. \\
\indent \textbf{Draw:} Player1 and Player2 are in the Explosion Area, or the round-timer is 0. \\
\indent Press ``q'' to quit directly.

\end{document}

